\documentclass{beamer}

\usepackage{amsmath}
\usepackage{amsfonts}
\usepackage{amssymb}


\usepackage{xcolor}
\usepackage{lmodern}


\usepackage[spanish]{babel}
\usepackage[utf8]{inputenc}


\title{Ejemplo de presentación}
\author{Fernando Chamizo}
\institute[UAM]{Universidad Autónoma de Madrid}
\date{10 de diciembre de 2021}


\usetheme{Madrid}
\usecolortheme{rose}


\begin{document}

%%%%%%%%%%%%%%%
% DIAPOSITIVA %
%%%%%%%%%%%%%%%
%crea la página del título
\begin{frame}
\titlepage
\end{frame}


%%%%%%%%%%%%%%%
% DIAPOSITIVA %
%%%%%%%%%%%%%%%
\begin{frame}
\frametitle{Tabla de contenidos}
\tableofcontents
\end{frame}


\section{Principio}
\begin{frame}
\frametitle{La diapositiva}
Mi primera diapositiva
\end{frame}

\begin{frame}
\frametitle{Bloques}
Texto
\uncover<2>{\begin{block}{Un bloque}
Solo en la segunda pulsación
\end{block}}
\uncover<3->{\begin{block}{}
Un bloque sin cabecera a partir de la tercera.
\end{block}}

\uncover<3-4>{
{\setbeamercolor{block title}{bg=green, fg=red}
\setbeamercolor{block body}{bg=cyan!10, fg=blue}
\begin{block}{Hola}
En la tercera y cuarta con otros colores.
\end{block}}
}

\begin{itemize}
 \item<5-> Desde la quinta
 \item<6> Sexta
\end{itemize}



\end{frame}


\section{Fin}
\begin{frame}
\begin{center}
 \Huge Gracias por su atención
\end{center}
\end{frame}


\end{document}

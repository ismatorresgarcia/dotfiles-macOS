\documentclass[14pt]{beamer}
\usetheme{Copenhagen}
%\usecolortheme{beaver}

%\setbeamertemplate{navigation symbols}{} %la barra de navegación la oculto
%\setbeamercovered{transparent}{} %veo lo que viene en la sig. diapo
%\setbeamertemplate{healine}{} %oculta la barra de nav. de arriba de las secciones

\usepackage[T1]{fontenc}
\usepackage[spanish]{babel}
\usepackage{amsmath, amsfonts, amsthm}
%\usepackage{pgfpages} %para imprimir las diapos si quieres
%\pgfpagesuselayout{4 on 1}[a4paper, border, shrink=5mm]

\title{Plantilla Beamer}
\author{Ismael Torres García}
\date{Agosto de 2022}

\begin{document}

\maketitle

\begin{frame}{Índice}
    \tableofcontents
\end{frame}

\section{Intro}

\begin{frame}{Supercosa 1}

\begin{itemize}
    \item<1-> Cosa 1
    \item<2-> Cosa 2
    \item<3-> Cosa 3
\end{itemize}

\only<2>{Einstein} %con 'only' la palabra 'Dirac' de después no sabe que 'Einstein' está ahí
\onslide<3>{Dirac}

Puedo poner \alert<1>{alertas}. También puedo poner las cosas en \textbf<2>{negrita}.

\end{frame}

\section{Matemáticas}

\begin{frame}{Cosas especiales}

\begin{block}{Coso 1}
    Un texto
\end{block}

\begin{exampleblock}{Ejemplo 1}
     Un ejemplo
\end{exampleblock}

\begin{block}{Teorema (Pitágoras)}
    $a^2+b^2=c^2$
\end{block}

\begin{proof}<2>
    Te lo dejo a ti
\end{proof}
    
\end{frame}

\begin{frame}{Con dos columnas}
    \begin{columns}
        \column{0.5\textwidth} blibli bliblbiblbib. blibli bliblbiblbib. blibli bliblbiblbib.blibli bliblbiblbib.blibli bliblbiblbib 
        \column{0.5\textwidth} blibli bliblbiblbib.blibli bliblbiblbibblibli bliblbiblbib.blibli bliblbiblbibblibli bliblbiblbib.blibli bliblbiblbib.
    \end{columns}
\end{frame}

\section{Prueba}

\begin{frame}{Otra cosa}
    Más cosas
\end{frame}

\subsection{Otra prueba}

\begin{frame}{Otra cosa más}
    Y más cosas
\end{frame}
    
\end{document}
